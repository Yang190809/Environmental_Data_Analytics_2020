\documentclass[]{article}
\usepackage{lmodern}
\usepackage{amssymb,amsmath}
\usepackage{ifxetex,ifluatex}
\usepackage{fixltx2e} % provides \textsubscript
\ifnum 0\ifxetex 1\fi\ifluatex 1\fi=0 % if pdftex
  \usepackage[T1]{fontenc}
  \usepackage[utf8]{inputenc}
\else % if luatex or xelatex
  \ifxetex
    \usepackage{mathspec}
  \else
    \usepackage{fontspec}
  \fi
  \defaultfontfeatures{Ligatures=TeX,Scale=MatchLowercase}
\fi
% use upquote if available, for straight quotes in verbatim environments
\IfFileExists{upquote.sty}{\usepackage{upquote}}{}
% use microtype if available
\IfFileExists{microtype.sty}{%
\usepackage{microtype}
\UseMicrotypeSet[protrusion]{basicmath} % disable protrusion for tt fonts
}{}
\usepackage[margin=2.54cm]{geometry}
\usepackage{hyperref}
\hypersetup{unicode=true,
            pdftitle={Assignment 4: Data Wrangling},
            pdfauthor={Yang Wang},
            pdfborder={0 0 0},
            breaklinks=true}
\urlstyle{same}  % don't use monospace font for urls
\usepackage{color}
\usepackage{fancyvrb}
\newcommand{\VerbBar}{|}
\newcommand{\VERB}{\Verb[commandchars=\\\{\}]}
\DefineVerbatimEnvironment{Highlighting}{Verbatim}{commandchars=\\\{\}}
% Add ',fontsize=\small' for more characters per line
\usepackage{framed}
\definecolor{shadecolor}{RGB}{248,248,248}
\newenvironment{Shaded}{\begin{snugshade}}{\end{snugshade}}
\newcommand{\AlertTok}[1]{\textcolor[rgb]{0.94,0.16,0.16}{#1}}
\newcommand{\AnnotationTok}[1]{\textcolor[rgb]{0.56,0.35,0.01}{\textbf{\textit{#1}}}}
\newcommand{\AttributeTok}[1]{\textcolor[rgb]{0.77,0.63,0.00}{#1}}
\newcommand{\BaseNTok}[1]{\textcolor[rgb]{0.00,0.00,0.81}{#1}}
\newcommand{\BuiltInTok}[1]{#1}
\newcommand{\CharTok}[1]{\textcolor[rgb]{0.31,0.60,0.02}{#1}}
\newcommand{\CommentTok}[1]{\textcolor[rgb]{0.56,0.35,0.01}{\textit{#1}}}
\newcommand{\CommentVarTok}[1]{\textcolor[rgb]{0.56,0.35,0.01}{\textbf{\textit{#1}}}}
\newcommand{\ConstantTok}[1]{\textcolor[rgb]{0.00,0.00,0.00}{#1}}
\newcommand{\ControlFlowTok}[1]{\textcolor[rgb]{0.13,0.29,0.53}{\textbf{#1}}}
\newcommand{\DataTypeTok}[1]{\textcolor[rgb]{0.13,0.29,0.53}{#1}}
\newcommand{\DecValTok}[1]{\textcolor[rgb]{0.00,0.00,0.81}{#1}}
\newcommand{\DocumentationTok}[1]{\textcolor[rgb]{0.56,0.35,0.01}{\textbf{\textit{#1}}}}
\newcommand{\ErrorTok}[1]{\textcolor[rgb]{0.64,0.00,0.00}{\textbf{#1}}}
\newcommand{\ExtensionTok}[1]{#1}
\newcommand{\FloatTok}[1]{\textcolor[rgb]{0.00,0.00,0.81}{#1}}
\newcommand{\FunctionTok}[1]{\textcolor[rgb]{0.00,0.00,0.00}{#1}}
\newcommand{\ImportTok}[1]{#1}
\newcommand{\InformationTok}[1]{\textcolor[rgb]{0.56,0.35,0.01}{\textbf{\textit{#1}}}}
\newcommand{\KeywordTok}[1]{\textcolor[rgb]{0.13,0.29,0.53}{\textbf{#1}}}
\newcommand{\NormalTok}[1]{#1}
\newcommand{\OperatorTok}[1]{\textcolor[rgb]{0.81,0.36,0.00}{\textbf{#1}}}
\newcommand{\OtherTok}[1]{\textcolor[rgb]{0.56,0.35,0.01}{#1}}
\newcommand{\PreprocessorTok}[1]{\textcolor[rgb]{0.56,0.35,0.01}{\textit{#1}}}
\newcommand{\RegionMarkerTok}[1]{#1}
\newcommand{\SpecialCharTok}[1]{\textcolor[rgb]{0.00,0.00,0.00}{#1}}
\newcommand{\SpecialStringTok}[1]{\textcolor[rgb]{0.31,0.60,0.02}{#1}}
\newcommand{\StringTok}[1]{\textcolor[rgb]{0.31,0.60,0.02}{#1}}
\newcommand{\VariableTok}[1]{\textcolor[rgb]{0.00,0.00,0.00}{#1}}
\newcommand{\VerbatimStringTok}[1]{\textcolor[rgb]{0.31,0.60,0.02}{#1}}
\newcommand{\WarningTok}[1]{\textcolor[rgb]{0.56,0.35,0.01}{\textbf{\textit{#1}}}}
\usepackage{graphicx,grffile}
\makeatletter
\def\maxwidth{\ifdim\Gin@nat@width>\linewidth\linewidth\else\Gin@nat@width\fi}
\def\maxheight{\ifdim\Gin@nat@height>\textheight\textheight\else\Gin@nat@height\fi}
\makeatother
% Scale images if necessary, so that they will not overflow the page
% margins by default, and it is still possible to overwrite the defaults
% using explicit options in \includegraphics[width, height, ...]{}
\setkeys{Gin}{width=\maxwidth,height=\maxheight,keepaspectratio}
\IfFileExists{parskip.sty}{%
\usepackage{parskip}
}{% else
\setlength{\parindent}{0pt}
\setlength{\parskip}{6pt plus 2pt minus 1pt}
}
\setlength{\emergencystretch}{3em}  % prevent overfull lines
\providecommand{\tightlist}{%
  \setlength{\itemsep}{0pt}\setlength{\parskip}{0pt}}
\setcounter{secnumdepth}{0}
% Redefines (sub)paragraphs to behave more like sections
\ifx\paragraph\undefined\else
\let\oldparagraph\paragraph
\renewcommand{\paragraph}[1]{\oldparagraph{#1}\mbox{}}
\fi
\ifx\subparagraph\undefined\else
\let\oldsubparagraph\subparagraph
\renewcommand{\subparagraph}[1]{\oldsubparagraph{#1}\mbox{}}
\fi

%%% Use protect on footnotes to avoid problems with footnotes in titles
\let\rmarkdownfootnote\footnote%
\def\footnote{\protect\rmarkdownfootnote}

%%% Change title format to be more compact
\usepackage{titling}

% Create subtitle command for use in maketitle
\providecommand{\subtitle}[1]{
  \posttitle{
    \begin{center}\large#1\end{center}
    }
}

\setlength{\droptitle}{-2em}

  \title{Assignment 4: Data Wrangling}
    \pretitle{\vspace{\droptitle}\centering\huge}
  \posttitle{\par}
    \author{Yang Wang}
    \preauthor{\centering\large\emph}
  \postauthor{\par}
    \date{}
    \predate{}\postdate{}
  

\begin{document}
\maketitle

\hypertarget{overview}{%
\subsection{OVERVIEW}\label{overview}}

This exercise accompanies the lessons in Environmental Data Analytics on
Data Wrangling

\hypertarget{directions}{%
\subsection{Directions}\label{directions}}

\begin{enumerate}
\def\labelenumi{\arabic{enumi}.}
\tightlist
\item
  Change ``Student Name'' on line 3 (above) with your name.
\item
  Work through the steps, \textbf{creating code and output} that fulfill
  each instruction.
\item
  Be sure to \textbf{answer the questions} in this assignment document.
\item
  When you have completed the assignment, \textbf{Knit} the text and
  code into a single PDF file.
\item
  After Knitting, submit the completed exercise (PDF file) to the
  dropbox in Sakai. Add your last name into the file name (e.g.,
  ``Salk\_A04\_DataWrangling.Rmd'') prior to submission.
\end{enumerate}

The completed exercise is due on Tuesday, February 4 at 1:00 pm.

\hypertarget{set-up-your-session}{%
\subsection{Set up your session}\label{set-up-your-session}}

\begin{enumerate}
\def\labelenumi{\arabic{enumi}.}
\item
  Check your working directory, load the \texttt{tidyverse} and
  \texttt{lubridate} packages, and upload all four raw data files
  associated with the EPA Air dataset. See the README file for the EPA
  air datasets for more information (especially if you have not worked
  with air quality data previously).
\item
  Explore the dimensions, column names, and structure of the datasets.
\end{enumerate}

\begin{Shaded}
\begin{Highlighting}[]
\CommentTok{#1}
\KeywordTok{getwd}\NormalTok{()}
\end{Highlighting}
\end{Shaded}

\begin{verbatim}
## [1] "C:/Users/26059/OneDrive/Desktop/ENV 872 R/Yang_ENV872"
\end{verbatim}

\begin{Shaded}
\begin{Highlighting}[]
\KeywordTok{library}\NormalTok{(tidyverse)}
\KeywordTok{library}\NormalTok{(lubridate)}
\NormalTok{air1<-}\StringTok{ }\KeywordTok{read.csv}\NormalTok{(}\StringTok{"./Data/Raw/EPAair_O3_NC2018_raw.csv"}\NormalTok{)}
\NormalTok{air2<-}\StringTok{ }\KeywordTok{read.csv}\NormalTok{(}\StringTok{"./Data/Raw/EPAair_O3_NC2019_raw.csv"}\NormalTok{)}
\NormalTok{air3<-}\StringTok{ }\KeywordTok{read.csv}\NormalTok{(}\StringTok{"./Data/Raw/EPAair_PM25_NC2018_raw.csv"}\NormalTok{)}
\NormalTok{air4<-}\StringTok{ }\KeywordTok{read.csv}\NormalTok{(}\StringTok{"./Data/Raw/EPAair_PM25_NC2019_raw.csv"}\NormalTok{)}

\CommentTok{#2}
\KeywordTok{dim}\NormalTok{(air1)}
\end{Highlighting}
\end{Shaded}

\begin{verbatim}
## [1] 9737   20
\end{verbatim}

\begin{Shaded}
\begin{Highlighting}[]
\KeywordTok{dim}\NormalTok{(air2)}
\end{Highlighting}
\end{Shaded}

\begin{verbatim}
## [1] 10592    20
\end{verbatim}

\begin{Shaded}
\begin{Highlighting}[]
\KeywordTok{dim}\NormalTok{(air3)}
\end{Highlighting}
\end{Shaded}

\begin{verbatim}
## [1] 8983   20
\end{verbatim}

\begin{Shaded}
\begin{Highlighting}[]
\KeywordTok{dim}\NormalTok{(air4)}
\end{Highlighting}
\end{Shaded}

\begin{verbatim}
## [1] 8581   20
\end{verbatim}

\begin{Shaded}
\begin{Highlighting}[]
\KeywordTok{head}\NormalTok{(air1)}
\end{Highlighting}
\end{Shaded}

\begin{verbatim}
##         Date Source   Site.ID POC Daily.Max.8.hour.Ozone.Concentration
## 1 03/01/2018    AQS 370030005   1                                0.043
## 2 03/02/2018    AQS 370030005   1                                0.046
## 3 03/03/2018    AQS 370030005   1                                0.047
## 4 03/04/2018    AQS 370030005   1                                0.049
## 5 03/05/2018    AQS 370030005   1                                0.047
## 6 03/06/2018    AQS 370030005   1                                0.030
##   UNITS DAILY_AQI_VALUE             Site.Name DAILY_OBS_COUNT
## 1   ppm              40 Taylorsville Liledoun              17
## 2   ppm              43 Taylorsville Liledoun              17
## 3   ppm              44 Taylorsville Liledoun              17
## 4   ppm              45 Taylorsville Liledoun              17
## 5   ppm              44 Taylorsville Liledoun              17
## 6   ppm              28 Taylorsville Liledoun              17
##   PERCENT_COMPLETE AQS_PARAMETER_CODE AQS_PARAMETER_DESC CBSA_CODE
## 1              100              44201              Ozone     25860
## 2              100              44201              Ozone     25860
## 3              100              44201              Ozone     25860
## 4              100              44201              Ozone     25860
## 5              100              44201              Ozone     25860
## 6              100              44201              Ozone     25860
##                      CBSA_NAME STATE_CODE          STATE COUNTY_CODE
## 1 Hickory-Lenoir-Morganton, NC         37 North Carolina           3
## 2 Hickory-Lenoir-Morganton, NC         37 North Carolina           3
## 3 Hickory-Lenoir-Morganton, NC         37 North Carolina           3
## 4 Hickory-Lenoir-Morganton, NC         37 North Carolina           3
## 5 Hickory-Lenoir-Morganton, NC         37 North Carolina           3
## 6 Hickory-Lenoir-Morganton, NC         37 North Carolina           3
##      COUNTY SITE_LATITUDE SITE_LONGITUDE
## 1 Alexander       35.9138        -81.191
## 2 Alexander       35.9138        -81.191
## 3 Alexander       35.9138        -81.191
## 4 Alexander       35.9138        -81.191
## 5 Alexander       35.9138        -81.191
## 6 Alexander       35.9138        -81.191
\end{verbatim}

\begin{Shaded}
\begin{Highlighting}[]
\KeywordTok{head}\NormalTok{(air2)}
\end{Highlighting}
\end{Shaded}

\begin{verbatim}
##         Date Source   Site.ID POC Daily.Max.8.hour.Ozone.Concentration
## 1 01/01/2019 AirNow 370030005   1                                0.029
## 2 01/02/2019 AirNow 370030005   1                                0.018
## 3 01/03/2019 AirNow 370030005   1                                0.016
## 4 01/04/2019 AirNow 370030005   1                                0.022
## 5 01/05/2019 AirNow 370030005   1                                0.037
## 6 01/06/2019 AirNow 370030005   1                                0.037
##   UNITS DAILY_AQI_VALUE             Site.Name DAILY_OBS_COUNT
## 1   ppm              27 Taylorsville Liledoun              24
## 2   ppm              17 Taylorsville Liledoun              24
## 3   ppm              15 Taylorsville Liledoun              24
## 4   ppm              20 Taylorsville Liledoun              24
## 5   ppm              34 Taylorsville Liledoun              24
## 6   ppm              34 Taylorsville Liledoun              24
##   PERCENT_COMPLETE AQS_PARAMETER_CODE AQS_PARAMETER_DESC CBSA_CODE
## 1              100              44201              Ozone     25860
## 2              100              44201              Ozone     25860
## 3              100              44201              Ozone     25860
## 4              100              44201              Ozone     25860
## 5              100              44201              Ozone     25860
## 6              100              44201              Ozone     25860
##                      CBSA_NAME STATE_CODE          STATE COUNTY_CODE
## 1 Hickory-Lenoir-Morganton, NC         37 North Carolina           3
## 2 Hickory-Lenoir-Morganton, NC         37 North Carolina           3
## 3 Hickory-Lenoir-Morganton, NC         37 North Carolina           3
## 4 Hickory-Lenoir-Morganton, NC         37 North Carolina           3
## 5 Hickory-Lenoir-Morganton, NC         37 North Carolina           3
## 6 Hickory-Lenoir-Morganton, NC         37 North Carolina           3
##      COUNTY SITE_LATITUDE SITE_LONGITUDE
## 1 Alexander       35.9138        -81.191
## 2 Alexander       35.9138        -81.191
## 3 Alexander       35.9138        -81.191
## 4 Alexander       35.9138        -81.191
## 5 Alexander       35.9138        -81.191
## 6 Alexander       35.9138        -81.191
\end{verbatim}

\begin{Shaded}
\begin{Highlighting}[]
\KeywordTok{head}\NormalTok{(air3)}
\end{Highlighting}
\end{Shaded}

\begin{verbatim}
##         Date Source   Site.ID POC Daily.Mean.PM2.5.Concentration    UNITS
## 1 01/02/2018    AQS 370110002   1                            2.9 ug/m3 LC
## 2 01/05/2018    AQS 370110002   1                            3.7 ug/m3 LC
## 3 01/08/2018    AQS 370110002   1                            5.3 ug/m3 LC
## 4 01/11/2018    AQS 370110002   1                            0.8 ug/m3 LC
## 5 01/14/2018    AQS 370110002   1                            2.5 ug/m3 LC
## 6 01/17/2018    AQS 370110002   1                            4.5 ug/m3 LC
##   DAILY_AQI_VALUE      Site.Name DAILY_OBS_COUNT PERCENT_COMPLETE
## 1              12 Linville Falls               1              100
## 2              15 Linville Falls               1              100
## 3              22 Linville Falls               1              100
## 4               3 Linville Falls               1              100
## 5              10 Linville Falls               1              100
## 6              19 Linville Falls               1              100
##   AQS_PARAMETER_CODE                     AQS_PARAMETER_DESC CBSA_CODE
## 1              88502 Acceptable PM2.5 AQI & Speciation Mass        NA
## 2              88502 Acceptable PM2.5 AQI & Speciation Mass        NA
## 3              88502 Acceptable PM2.5 AQI & Speciation Mass        NA
## 4              88502 Acceptable PM2.5 AQI & Speciation Mass        NA
## 5              88502 Acceptable PM2.5 AQI & Speciation Mass        NA
## 6              88502 Acceptable PM2.5 AQI & Speciation Mass        NA
##   CBSA_NAME STATE_CODE          STATE COUNTY_CODE COUNTY SITE_LATITUDE
## 1                   37 North Carolina          11  Avery      35.97235
## 2                   37 North Carolina          11  Avery      35.97235
## 3                   37 North Carolina          11  Avery      35.97235
## 4                   37 North Carolina          11  Avery      35.97235
## 5                   37 North Carolina          11  Avery      35.97235
## 6                   37 North Carolina          11  Avery      35.97235
##   SITE_LONGITUDE
## 1      -81.93307
## 2      -81.93307
## 3      -81.93307
## 4      -81.93307
## 5      -81.93307
## 6      -81.93307
\end{verbatim}

\begin{Shaded}
\begin{Highlighting}[]
\KeywordTok{head}\NormalTok{(air4)}
\end{Highlighting}
\end{Shaded}

\begin{verbatim}
##         Date Source   Site.ID POC Daily.Mean.PM2.5.Concentration    UNITS
## 1 01/03/2019    AQS 370110002   1                            1.6 ug/m3 LC
## 2 01/06/2019    AQS 370110002   1                            1.0 ug/m3 LC
## 3 01/09/2019    AQS 370110002   1                            1.3 ug/m3 LC
## 4 01/12/2019    AQS 370110002   1                            6.3 ug/m3 LC
## 5 01/15/2019    AQS 370110002   1                            2.6 ug/m3 LC
## 6 01/18/2019    AQS 370110002   1                            1.2 ug/m3 LC
##   DAILY_AQI_VALUE      Site.Name DAILY_OBS_COUNT PERCENT_COMPLETE
## 1               7 Linville Falls               1              100
## 2               4 Linville Falls               1              100
## 3               5 Linville Falls               1              100
## 4              26 Linville Falls               1              100
## 5              11 Linville Falls               1              100
## 6               5 Linville Falls               1              100
##   AQS_PARAMETER_CODE                     AQS_PARAMETER_DESC CBSA_CODE
## 1              88502 Acceptable PM2.5 AQI & Speciation Mass        NA
## 2              88502 Acceptable PM2.5 AQI & Speciation Mass        NA
## 3              88502 Acceptable PM2.5 AQI & Speciation Mass        NA
## 4              88502 Acceptable PM2.5 AQI & Speciation Mass        NA
## 5              88502 Acceptable PM2.5 AQI & Speciation Mass        NA
## 6              88502 Acceptable PM2.5 AQI & Speciation Mass        NA
##   CBSA_NAME STATE_CODE          STATE COUNTY_CODE COUNTY SITE_LATITUDE
## 1                   37 North Carolina          11  Avery      35.97235
## 2                   37 North Carolina          11  Avery      35.97235
## 3                   37 North Carolina          11  Avery      35.97235
## 4                   37 North Carolina          11  Avery      35.97235
## 5                   37 North Carolina          11  Avery      35.97235
## 6                   37 North Carolina          11  Avery      35.97235
##   SITE_LONGITUDE
## 1      -81.93307
## 2      -81.93307
## 3      -81.93307
## 4      -81.93307
## 5      -81.93307
## 6      -81.93307
\end{verbatim}

\begin{Shaded}
\begin{Highlighting}[]
\KeywordTok{str}\NormalTok{(air1)}
\end{Highlighting}
\end{Shaded}

\begin{verbatim}
## 'data.frame':    9737 obs. of  20 variables:
##  $ Date                                : Factor w/ 364 levels "01/01/2018","01/02/2018",..: 60 61 62 63 64 65 66 67 68 69 ...
##  $ Source                              : Factor w/ 1 level "AQS": 1 1 1 1 1 1 1 1 1 1 ...
##  $ Site.ID                             : int  370030005 370030005 370030005 370030005 370030005 370030005 370030005 370030005 370030005 370030005 ...
##  $ POC                                 : int  1 1 1 1 1 1 1 1 1 1 ...
##  $ Daily.Max.8.hour.Ozone.Concentration: num  0.043 0.046 0.047 0.049 0.047 0.03 0.036 0.044 0.049 0.043 ...
##  $ UNITS                               : Factor w/ 1 level "ppm": 1 1 1 1 1 1 1 1 1 1 ...
##  $ DAILY_AQI_VALUE                     : int  40 43 44 45 44 28 33 41 45 40 ...
##  $ Site.Name                           : Factor w/ 40 levels "","Beaufort",..: 35 35 35 35 35 35 35 35 35 35 ...
##  $ DAILY_OBS_COUNT                     : int  17 17 17 17 17 17 17 17 17 17 ...
##  $ PERCENT_COMPLETE                    : num  100 100 100 100 100 100 100 100 100 100 ...
##  $ AQS_PARAMETER_CODE                  : int  44201 44201 44201 44201 44201 44201 44201 44201 44201 44201 ...
##  $ AQS_PARAMETER_DESC                  : Factor w/ 1 level "Ozone": 1 1 1 1 1 1 1 1 1 1 ...
##  $ CBSA_CODE                           : int  25860 25860 25860 25860 25860 25860 25860 25860 25860 25860 ...
##  $ CBSA_NAME                           : Factor w/ 17 levels "","Asheville, NC",..: 9 9 9 9 9 9 9 9 9 9 ...
##  $ STATE_CODE                          : int  37 37 37 37 37 37 37 37 37 37 ...
##  $ STATE                               : Factor w/ 1 level "North Carolina": 1 1 1 1 1 1 1 1 1 1 ...
##  $ COUNTY_CODE                         : int  3 3 3 3 3 3 3 3 3 3 ...
##  $ COUNTY                              : Factor w/ 32 levels "Alexander","Avery",..: 1 1 1 1 1 1 1 1 1 1 ...
##  $ SITE_LATITUDE                       : num  35.9 35.9 35.9 35.9 35.9 ...
##  $ SITE_LONGITUDE                      : num  -81.2 -81.2 -81.2 -81.2 -81.2 ...
\end{verbatim}

\begin{Shaded}
\begin{Highlighting}[]
\KeywordTok{str}\NormalTok{(air2)}
\end{Highlighting}
\end{Shaded}

\begin{verbatim}
## 'data.frame':    10592 obs. of  20 variables:
##  $ Date                                : Factor w/ 365 levels "01/01/2019","01/02/2019",..: 1 2 3 4 5 6 7 8 9 10 ...
##  $ Source                              : Factor w/ 2 levels "AirNow","AQS": 1 1 1 1 1 1 1 1 1 1 ...
##  $ Site.ID                             : int  370030005 370030005 370030005 370030005 370030005 370030005 370030005 370030005 370030005 370030005 ...
##  $ POC                                 : int  1 1 1 1 1 1 1 1 1 1 ...
##  $ Daily.Max.8.hour.Ozone.Concentration: num  0.029 0.018 0.016 0.022 0.037 0.037 0.029 0.038 0.038 0.03 ...
##  $ UNITS                               : Factor w/ 1 level "ppm": 1 1 1 1 1 1 1 1 1 1 ...
##  $ DAILY_AQI_VALUE                     : int  27 17 15 20 34 34 27 35 35 28 ...
##  $ Site.Name                           : Factor w/ 38 levels "","Beaufort",..: 33 33 33 33 33 33 33 33 33 33 ...
##  $ DAILY_OBS_COUNT                     : int  24 24 24 24 24 24 24 24 24 24 ...
##  $ PERCENT_COMPLETE                    : num  100 100 100 100 100 100 100 100 100 100 ...
##  $ AQS_PARAMETER_CODE                  : int  44201 44201 44201 44201 44201 44201 44201 44201 44201 44201 ...
##  $ AQS_PARAMETER_DESC                  : Factor w/ 1 level "Ozone": 1 1 1 1 1 1 1 1 1 1 ...
##  $ CBSA_CODE                           : int  25860 25860 25860 25860 25860 25860 25860 25860 25860 25860 ...
##  $ CBSA_NAME                           : Factor w/ 15 levels "","Asheville, NC",..: 8 8 8 8 8 8 8 8 8 8 ...
##  $ STATE_CODE                          : int  37 37 37 37 37 37 37 37 37 37 ...
##  $ STATE                               : Factor w/ 1 level "North Carolina": 1 1 1 1 1 1 1 1 1 1 ...
##  $ COUNTY_CODE                         : int  3 3 3 3 3 3 3 3 3 3 ...
##  $ COUNTY                              : Factor w/ 30 levels "Alexander","Avery",..: 1 1 1 1 1 1 1 1 1 1 ...
##  $ SITE_LATITUDE                       : num  35.9 35.9 35.9 35.9 35.9 ...
##  $ SITE_LONGITUDE                      : num  -81.2 -81.2 -81.2 -81.2 -81.2 ...
\end{verbatim}

\begin{Shaded}
\begin{Highlighting}[]
\KeywordTok{str}\NormalTok{(air3)}
\end{Highlighting}
\end{Shaded}

\begin{verbatim}
## 'data.frame':    8983 obs. of  20 variables:
##  $ Date                          : Factor w/ 365 levels "01/01/2018","01/02/2018",..: 2 5 8 11 14 17 20 23 26 29 ...
##  $ Source                        : Factor w/ 1 level "AQS": 1 1 1 1 1 1 1 1 1 1 ...
##  $ Site.ID                       : int  370110002 370110002 370110002 370110002 370110002 370110002 370110002 370110002 370110002 370110002 ...
##  $ POC                           : int  1 1 1 1 1 1 1 1 1 1 ...
##  $ Daily.Mean.PM2.5.Concentration: num  2.9 3.7 5.3 0.8 2.5 4.5 1.8 2.5 4.2 1.7 ...
##  $ UNITS                         : Factor w/ 1 level "ug/m3 LC": 1 1 1 1 1 1 1 1 1 1 ...
##  $ DAILY_AQI_VALUE               : int  12 15 22 3 10 19 8 10 18 7 ...
##  $ Site.Name                     : Factor w/ 25 levels "","Blackstone",..: 15 15 15 15 15 15 15 15 15 15 ...
##  $ DAILY_OBS_COUNT               : int  1 1 1 1 1 1 1 1 1 1 ...
##  $ PERCENT_COMPLETE              : num  100 100 100 100 100 100 100 100 100 100 ...
##  $ AQS_PARAMETER_CODE            : int  88502 88502 88502 88502 88502 88502 88502 88502 88502 88502 ...
##  $ AQS_PARAMETER_DESC            : Factor w/ 2 levels "Acceptable PM2.5 AQI & Speciation Mass",..: 1 1 1 1 1 1 1 1 1 1 ...
##  $ CBSA_CODE                     : int  NA NA NA NA NA NA NA NA NA NA ...
##  $ CBSA_NAME                     : Factor w/ 14 levels "","Asheville, NC",..: 1 1 1 1 1 1 1 1 1 1 ...
##  $ STATE_CODE                    : int  37 37 37 37 37 37 37 37 37 37 ...
##  $ STATE                         : Factor w/ 1 level "North Carolina": 1 1 1 1 1 1 1 1 1 1 ...
##  $ COUNTY_CODE                   : int  11 11 11 11 11 11 11 11 11 11 ...
##  $ COUNTY                        : Factor w/ 21 levels "Avery","Buncombe",..: 1 1 1 1 1 1 1 1 1 1 ...
##  $ SITE_LATITUDE                 : num  36 36 36 36 36 ...
##  $ SITE_LONGITUDE                : num  -81.9 -81.9 -81.9 -81.9 -81.9 ...
\end{verbatim}

\begin{Shaded}
\begin{Highlighting}[]
\KeywordTok{str}\NormalTok{(air4)}
\end{Highlighting}
\end{Shaded}

\begin{verbatim}
## 'data.frame':    8581 obs. of  20 variables:
##  $ Date                          : Factor w/ 365 levels "01/01/2019","01/02/2019",..: 3 6 9 12 15 18 21 24 27 30 ...
##  $ Source                        : Factor w/ 2 levels "AirNow","AQS": 2 2 2 2 2 2 2 2 2 2 ...
##  $ Site.ID                       : int  370110002 370110002 370110002 370110002 370110002 370110002 370110002 370110002 370110002 370110002 ...
##  $ POC                           : int  1 1 1 1 1 1 1 1 1 1 ...
##  $ Daily.Mean.PM2.5.Concentration: num  1.6 1 1.3 6.3 2.6 1.2 1.5 1.5 3.7 1.6 ...
##  $ UNITS                         : Factor w/ 1 level "ug/m3 LC": 1 1 1 1 1 1 1 1 1 1 ...
##  $ DAILY_AQI_VALUE               : int  7 4 5 26 11 5 6 6 15 7 ...
##  $ Site.Name                     : Factor w/ 25 levels "","Board Of Ed. Bldg.",..: 14 14 14 14 14 14 14 14 14 14 ...
##  $ DAILY_OBS_COUNT               : int  1 1 1 1 1 1 1 1 1 1 ...
##  $ PERCENT_COMPLETE              : num  100 100 100 100 100 100 100 100 100 100 ...
##  $ AQS_PARAMETER_CODE            : int  88502 88502 88502 88502 88502 88502 88502 88502 88502 88502 ...
##  $ AQS_PARAMETER_DESC            : Factor w/ 2 levels "Acceptable PM2.5 AQI & Speciation Mass",..: 1 1 1 1 1 1 1 1 1 1 ...
##  $ CBSA_CODE                     : int  NA NA NA NA NA NA NA NA NA NA ...
##  $ CBSA_NAME                     : Factor w/ 14 levels "","Asheville, NC",..: 1 1 1 1 1 1 1 1 1 1 ...
##  $ STATE_CODE                    : int  37 37 37 37 37 37 37 37 37 37 ...
##  $ STATE                         : Factor w/ 1 level "North Carolina": 1 1 1 1 1 1 1 1 1 1 ...
##  $ COUNTY_CODE                   : int  11 11 11 11 11 11 11 11 11 11 ...
##  $ COUNTY                        : Factor w/ 21 levels "Avery","Buncombe",..: 1 1 1 1 1 1 1 1 1 1 ...
##  $ SITE_LATITUDE                 : num  36 36 36 36 36 ...
##  $ SITE_LONGITUDE                : num  -81.9 -81.9 -81.9 -81.9 -81.9 ...
\end{verbatim}

\hypertarget{wrangle-individual-datasets-to-create-processed-files.}{%
\subsection{Wrangle individual datasets to create processed
files.}\label{wrangle-individual-datasets-to-create-processed-files.}}

\begin{enumerate}
\def\labelenumi{\arabic{enumi}.}
\setcounter{enumi}{2}
\tightlist
\item
  Change date to date
\item
  Select the following columns: Date, DAILY\_AQI\_VALUE, Site.Name,
  AQS\_PARAMETER\_DESC, COUNTY, SITE\_LATITUDE, SITE\_LONGITUDE
\item
  For the PM2.5 datasets, fill all cells in AQS\_PARAMETER\_DESC with
  ``PM2.5'' (all cells in this column should be identical).
\item
  Save all four processed datasets in the Processed folder. Use the same
  file names as the raw files but replace ``raw'' with ``processed''.
\end{enumerate}

\begin{Shaded}
\begin{Highlighting}[]
\CommentTok{#3}
\CommentTok{#class(air1$Date)}
\NormalTok{air1}\OperatorTok{$}\NormalTok{Date <-}\StringTok{ }\KeywordTok{as.Date}\NormalTok{(air1}\OperatorTok{$}\NormalTok{Date, }\DataTypeTok{format =} \StringTok{"%m/%d/%Y"}\NormalTok{)}
\CommentTok{#class(air1$Date)}
\NormalTok{air2}\OperatorTok{$}\NormalTok{Date <-}\StringTok{ }\KeywordTok{as.Date}\NormalTok{(air2}\OperatorTok{$}\NormalTok{Date, }\DataTypeTok{format =} \StringTok{"%m/%d/%Y"}\NormalTok{)}
\NormalTok{air3}\OperatorTok{$}\NormalTok{Date <-}\StringTok{ }\KeywordTok{as.Date}\NormalTok{(air3}\OperatorTok{$}\NormalTok{Date, }\DataTypeTok{format =} \StringTok{"%m/%d/%Y"}\NormalTok{)}
\NormalTok{air4}\OperatorTok{$}\NormalTok{Date <-}\StringTok{ }\KeywordTok{as.Date}\NormalTok{(air4}\OperatorTok{$}\NormalTok{Date, }\DataTypeTok{format =} \StringTok{"%m/%d/%Y"}\NormalTok{)}
\CommentTok{#4}
\NormalTok{air1}\FloatTok{.1}\NormalTok{<-}\StringTok{ }\KeywordTok{select}\NormalTok{(air1,Date, DAILY_AQI_VALUE, Site.Name, AQS_PARAMETER_DESC, COUNTY, SITE_LATITUDE, SITE_LONGITUDE)}
\NormalTok{air2}\FloatTok{.1}\NormalTok{<-}\StringTok{ }\KeywordTok{select}\NormalTok{(air2,Date, DAILY_AQI_VALUE, Site.Name, AQS_PARAMETER_DESC, COUNTY, SITE_LATITUDE, SITE_LONGITUDE)}
\NormalTok{air3}\FloatTok{.1}\NormalTok{<-}\StringTok{ }\KeywordTok{select}\NormalTok{(air3,Date, DAILY_AQI_VALUE, Site.Name, AQS_PARAMETER_DESC, COUNTY, SITE_LATITUDE, SITE_LONGITUDE)}
\NormalTok{air4}\FloatTok{.1}\NormalTok{<-}\StringTok{ }\KeywordTok{select}\NormalTok{(air4,Date, DAILY_AQI_VALUE, Site.Name, AQS_PARAMETER_DESC, COUNTY, SITE_LATITUDE, SITE_LONGITUDE)}
\CommentTok{#5}
\NormalTok{air3}\FloatTok{.1}\OperatorTok{$}\NormalTok{AQS_PARAMETER_DESC<-}\StringTok{ "PM2.5"}
\NormalTok{air4}\FloatTok{.1}\OperatorTok{$}\NormalTok{AQS_PARAMETER_DESC<-}\StringTok{ "PM2.5"}
\CommentTok{#6}
\KeywordTok{write.csv}\NormalTok{(air1}\FloatTok{.1}\NormalTok{, }\DataTypeTok{row.names =} \OtherTok{FALSE}\NormalTok{, }\DataTypeTok{file =} \StringTok{"./Data/Processed/EPAair_O3_NC2018_processed.csv"}\NormalTok{)}
\KeywordTok{write.csv}\NormalTok{(air2}\FloatTok{.1}\NormalTok{, }\DataTypeTok{row.names =} \OtherTok{FALSE}\NormalTok{, }\DataTypeTok{file =} \StringTok{"./Data/Processed/EPAair_O3_NC2019_processed.csv"}\NormalTok{)}
\KeywordTok{write.csv}\NormalTok{(air3}\FloatTok{.1}\NormalTok{, }\DataTypeTok{row.names =} \OtherTok{FALSE}\NormalTok{, }\DataTypeTok{file =} \StringTok{"./Data/Processed/EPAair_PM25_NC2018_processed.csv"}\NormalTok{)}
\KeywordTok{write.csv}\NormalTok{(air4}\FloatTok{.1}\NormalTok{, }\DataTypeTok{row.names =} \OtherTok{FALSE}\NormalTok{, }\DataTypeTok{file =} \StringTok{"./Data/Processed/EPAair_PM25_NC2019_processed.csv"}\NormalTok{)}
\end{Highlighting}
\end{Shaded}

\hypertarget{combine-datasets}{%
\subsection{Combine datasets}\label{combine-datasets}}

\begin{enumerate}
\def\labelenumi{\arabic{enumi}.}
\setcounter{enumi}{6}
\tightlist
\item
  Combine the four datasets with \texttt{rbind}. Make sure your column
  names are identical prior to running this code.
\item
  Wrangle your new dataset with a pipe function (\%\textgreater\%) so
  that it fills the following conditions:
\end{enumerate}

\begin{itemize}
\tightlist
\item
  Include all sites that the four data frames have in common: ``Linville
  Falls'', ``Durham Armory'', ``Leggett'', ``Hattie Avenue'', ``Clemmons
  Middle'', ``Mendenhall School'', ``Frying Pan Mountain'', ``West
  Johnston Co.'', ``Garinger High School'', ``Castle Hayne'', ``Pitt
  Agri. Center'', ``Bryson City'', ``Millbrook School'' (the function
  \texttt{intersect} can figure out common factor levels)
\item
  Some sites have multiple measurements per day. Use the
  split-apply-combine strategy to generate daily means: group by date,
  site, aqs parameter, and county. Take the mean of the AQI value,
  latitude, and longitude.
\item
  Add columns for ``Month'' and ``Year'' by parsing your ``Date'' column
  (hint: \texttt{lubridate} package)
\item
  Hint: the dimensions of this dataset should be 14,752 x 9.
\end{itemize}

\begin{enumerate}
\def\labelenumi{\arabic{enumi}.}
\setcounter{enumi}{8}
\tightlist
\item
  Spread your datasets such that AQI values for ozone and PM2.5 are in
  separate columns. Each location on a specific date should now occupy
  only one row.
\item
  Call up the dimensions of your new tidy dataset.
\item
  Save your processed dataset with the following file name:
  ``EPAair\_O3\_PM25\_NC1718\_Processed.csv''
\end{enumerate}

\begin{Shaded}
\begin{Highlighting}[]
\CommentTok{#7}

\NormalTok{air <-}\StringTok{ }\KeywordTok{rbind}\NormalTok{(air1}\FloatTok{.1}\NormalTok{, air2}\FloatTok{.1}\NormalTok{, air3}\FloatTok{.1}\NormalTok{, air4}\FloatTok{.1}\NormalTok{)}
\CommentTok{#summary(air$Site.Name)}

\CommentTok{#8}
\NormalTok{com.site.O3 <-}\StringTok{ }\KeywordTok{intersect}\NormalTok{(air1}\FloatTok{.1}\OperatorTok{$}\NormalTok{Site.Name, air2}\FloatTok{.1}\OperatorTok{$}\NormalTok{Site.Name)}
\NormalTok{com.site.PM2}\FloatTok{.5}\NormalTok{ <-}\StringTok{ }\KeywordTok{intersect}\NormalTok{(air3}\FloatTok{.1}\OperatorTok{$}\NormalTok{Site.Name, air4}\FloatTok{.1}\OperatorTok{$}\NormalTok{Site.Name)}
\NormalTok{com.site <-}\StringTok{ }\KeywordTok{intersect}\NormalTok{ (com.site.O3, com.site.PM2}\FloatTok{.5}\NormalTok{)}
\NormalTok{com.site <-}\StringTok{ }\NormalTok{com.site[}\OperatorTok{-}\DecValTok{13}\NormalTok{]}

\NormalTok{air.all <-}\StringTok{ }
\StringTok{  }\NormalTok{air}\OperatorTok
\StringTok{  }\KeywordTok{filter}\NormalTok{(Site.Name }\OperatorTok\StringTok{ }\NormalTok{com.site)}\OperatorTok
\StringTok{  }
\KeywordTok{group_by}\NormalTok{(Date, Site.Name, AQS_PARAMETER_DESC, COUNTY) }\OperatorTok
\KeywordTok{summarise}\NormalTok{(}\DataTypeTok{mean.AQI =} \KeywordTok{mean}\NormalTok{(DAILY_AQI_VALUE), }
            \DataTypeTok{mean.latitude =} \KeywordTok{mean}\NormalTok{(SITE_LATITUDE), }
          \DataTypeTok{mean.longitude =} \KeywordTok{mean}\NormalTok{(SITE_LONGITUDE))}\OperatorTok
\StringTok{  }
\KeywordTok{mutate}\NormalTok{( }\DataTypeTok{year=}\KeywordTok{year}\NormalTok{(Date))}\OperatorTok
\KeywordTok{mutate}\NormalTok{(}\DataTypeTok{month =} \KeywordTok{month}\NormalTok{(Date))}
\KeywordTok{dim}\NormalTok{(air.all)}
\end{Highlighting}
\end{Shaded}

\begin{verbatim}
## [1] 14752     9
\end{verbatim}

\begin{Shaded}
\begin{Highlighting}[]
\CommentTok{#9}
\NormalTok{air.spread <-}\StringTok{ }\KeywordTok{spread}\NormalTok{(air.all, AQS_PARAMETER_DESC,mean.AQI)}

\CommentTok{#10}
\KeywordTok{dim}\NormalTok{(air.spread)}
\end{Highlighting}
\end{Shaded}

\begin{verbatim}
## [1] 8976    9
\end{verbatim}

\begin{Shaded}
\begin{Highlighting}[]
\CommentTok{#11}
\KeywordTok{write.csv}\NormalTok{(air.spread, }\DataTypeTok{row.names =} \OtherTok{FALSE}\NormalTok{, }\DataTypeTok{file =} \StringTok{"./Data/Processed/EPAair_O3_PM25_NC1718_Processed.csv"}\NormalTok{)}
\end{Highlighting}
\end{Shaded}

\hypertarget{generate-summary-tables}{%
\subsection{Generate summary tables}\label{generate-summary-tables}}

\begin{enumerate}
\def\labelenumi{\arabic{enumi}.}
\setcounter{enumi}{11}
\item
  Use the split-apply-combine strategy to generate a summary data frame.
  Data should be grouped by site, month, and year. Generate the mean AQI
  values for ozone and PM2.5 for each group. Then, add a pipe to remove
  instances where a month and year are not available (use the function
  \texttt{drop\_na} in your pipe).
\item
  Call up the dimensions of the summary dataset.
\end{enumerate}

\begin{Shaded}
\begin{Highlighting}[]
\CommentTok{#12a}

\NormalTok{EPA.air <-}\StringTok{ }
\StringTok{  }\NormalTok{air.spread}\OperatorTok
\KeywordTok{group_by}\NormalTok{(Site.Name, year,month) }\OperatorTok
\KeywordTok{summarise}\NormalTok{(}\DataTypeTok{mean.Ozone =} \KeywordTok{mean}\NormalTok{(Ozone), }
            \DataTypeTok{mean.PM2.5 =} \KeywordTok{mean}\NormalTok{(PM2}\FloatTok{.5}\NormalTok{))}\OperatorTok
\StringTok{  }\KeywordTok{drop_na}\NormalTok{(year, month)}
\CommentTok{#12b}

\CommentTok{#13}
\KeywordTok{dim}\NormalTok{(EPA.air)}
\end{Highlighting}
\end{Shaded}

\begin{verbatim}
## [1] 308   5
\end{verbatim}

\begin{enumerate}
\def\labelenumi{\arabic{enumi}.}
\setcounter{enumi}{13}
\tightlist
\item
  Why did we use the function \texttt{drop\_na} rather than
  \texttt{na.omit}?
\end{enumerate}

\begin{quote}
Answer: Because we don't want to remove the NA in the columns containing
mean of ozone and PM2.5
\end{quote}


\end{document}
