\documentclass[]{article}
\usepackage{lmodern}
\usepackage{amssymb,amsmath}
\usepackage{ifxetex,ifluatex}
\usepackage{fixltx2e} % provides \textsubscript
\ifnum 0\ifxetex 1\fi\ifluatex 1\fi=0 % if pdftex
  \usepackage[T1]{fontenc}
  \usepackage[utf8]{inputenc}
\else % if luatex or xelatex
  \ifxetex
    \usepackage{mathspec}
  \else
    \usepackage{fontspec}
  \fi
  \defaultfontfeatures{Ligatures=TeX,Scale=MatchLowercase}
\fi
% use upquote if available, for straight quotes in verbatim environments
\IfFileExists{upquote.sty}{\usepackage{upquote}}{}
% use microtype if available
\IfFileExists{microtype.sty}{%
\usepackage{microtype}
\UseMicrotypeSet[protrusion]{basicmath} % disable protrusion for tt fonts
}{}
\usepackage[margin=2.54cm]{geometry}
\usepackage{hyperref}
\hypersetup{unicode=true,
            pdftitle={Assignment 3: Data Exploration},
            pdfauthor={Yang Wang},
            pdfborder={0 0 0},
            breaklinks=true}
\urlstyle{same}  % don't use monospace font for urls
\usepackage{color}
\usepackage{fancyvrb}
\newcommand{\VerbBar}{|}
\newcommand{\VERB}{\Verb[commandchars=\\\{\}]}
\DefineVerbatimEnvironment{Highlighting}{Verbatim}{commandchars=\\\{\}}
% Add ',fontsize=\small' for more characters per line
\usepackage{framed}
\definecolor{shadecolor}{RGB}{248,248,248}
\newenvironment{Shaded}{\begin{snugshade}}{\end{snugshade}}
\newcommand{\AlertTok}[1]{\textcolor[rgb]{0.94,0.16,0.16}{#1}}
\newcommand{\AnnotationTok}[1]{\textcolor[rgb]{0.56,0.35,0.01}{\textbf{\textit{#1}}}}
\newcommand{\AttributeTok}[1]{\textcolor[rgb]{0.77,0.63,0.00}{#1}}
\newcommand{\BaseNTok}[1]{\textcolor[rgb]{0.00,0.00,0.81}{#1}}
\newcommand{\BuiltInTok}[1]{#1}
\newcommand{\CharTok}[1]{\textcolor[rgb]{0.31,0.60,0.02}{#1}}
\newcommand{\CommentTok}[1]{\textcolor[rgb]{0.56,0.35,0.01}{\textit{#1}}}
\newcommand{\CommentVarTok}[1]{\textcolor[rgb]{0.56,0.35,0.01}{\textbf{\textit{#1}}}}
\newcommand{\ConstantTok}[1]{\textcolor[rgb]{0.00,0.00,0.00}{#1}}
\newcommand{\ControlFlowTok}[1]{\textcolor[rgb]{0.13,0.29,0.53}{\textbf{#1}}}
\newcommand{\DataTypeTok}[1]{\textcolor[rgb]{0.13,0.29,0.53}{#1}}
\newcommand{\DecValTok}[1]{\textcolor[rgb]{0.00,0.00,0.81}{#1}}
\newcommand{\DocumentationTok}[1]{\textcolor[rgb]{0.56,0.35,0.01}{\textbf{\textit{#1}}}}
\newcommand{\ErrorTok}[1]{\textcolor[rgb]{0.64,0.00,0.00}{\textbf{#1}}}
\newcommand{\ExtensionTok}[1]{#1}
\newcommand{\FloatTok}[1]{\textcolor[rgb]{0.00,0.00,0.81}{#1}}
\newcommand{\FunctionTok}[1]{\textcolor[rgb]{0.00,0.00,0.00}{#1}}
\newcommand{\ImportTok}[1]{#1}
\newcommand{\InformationTok}[1]{\textcolor[rgb]{0.56,0.35,0.01}{\textbf{\textit{#1}}}}
\newcommand{\KeywordTok}[1]{\textcolor[rgb]{0.13,0.29,0.53}{\textbf{#1}}}
\newcommand{\NormalTok}[1]{#1}
\newcommand{\OperatorTok}[1]{\textcolor[rgb]{0.81,0.36,0.00}{\textbf{#1}}}
\newcommand{\OtherTok}[1]{\textcolor[rgb]{0.56,0.35,0.01}{#1}}
\newcommand{\PreprocessorTok}[1]{\textcolor[rgb]{0.56,0.35,0.01}{\textit{#1}}}
\newcommand{\RegionMarkerTok}[1]{#1}
\newcommand{\SpecialCharTok}[1]{\textcolor[rgb]{0.00,0.00,0.00}{#1}}
\newcommand{\SpecialStringTok}[1]{\textcolor[rgb]{0.31,0.60,0.02}{#1}}
\newcommand{\StringTok}[1]{\textcolor[rgb]{0.31,0.60,0.02}{#1}}
\newcommand{\VariableTok}[1]{\textcolor[rgb]{0.00,0.00,0.00}{#1}}
\newcommand{\VerbatimStringTok}[1]{\textcolor[rgb]{0.31,0.60,0.02}{#1}}
\newcommand{\WarningTok}[1]{\textcolor[rgb]{0.56,0.35,0.01}{\textbf{\textit{#1}}}}
\usepackage{graphicx,grffile}
\makeatletter
\def\maxwidth{\ifdim\Gin@nat@width>\linewidth\linewidth\else\Gin@nat@width\fi}
\def\maxheight{\ifdim\Gin@nat@height>\textheight\textheight\else\Gin@nat@height\fi}
\makeatother
% Scale images if necessary, so that they will not overflow the page
% margins by default, and it is still possible to overwrite the defaults
% using explicit options in \includegraphics[width, height, ...]{}
\setkeys{Gin}{width=\maxwidth,height=\maxheight,keepaspectratio}
\IfFileExists{parskip.sty}{%
\usepackage{parskip}
}{% else
\setlength{\parindent}{0pt}
\setlength{\parskip}{6pt plus 2pt minus 1pt}
}
\setlength{\emergencystretch}{3em}  % prevent overfull lines
\providecommand{\tightlist}{%
  \setlength{\itemsep}{0pt}\setlength{\parskip}{0pt}}
\setcounter{secnumdepth}{0}
% Redefines (sub)paragraphs to behave more like sections
\ifx\paragraph\undefined\else
\let\oldparagraph\paragraph
\renewcommand{\paragraph}[1]{\oldparagraph{#1}\mbox{}}
\fi
\ifx\subparagraph\undefined\else
\let\oldsubparagraph\subparagraph
\renewcommand{\subparagraph}[1]{\oldsubparagraph{#1}\mbox{}}
\fi

%%% Use protect on footnotes to avoid problems with footnotes in titles
\let\rmarkdownfootnote\footnote%
\def\footnote{\protect\rmarkdownfootnote}

%%% Change title format to be more compact
\usepackage{titling}

% Create subtitle command for use in maketitle
\providecommand{\subtitle}[1]{
  \posttitle{
    \begin{center}\large#1\end{center}
    }
}

\setlength{\droptitle}{-2em}

  \title{Assignment 3: Data Exploration}
    \pretitle{\vspace{\droptitle}\centering\huge}
  \posttitle{\par}
    \author{Yang Wang}
    \preauthor{\centering\large\emph}
  \postauthor{\par}
    \date{}
    \predate{}\postdate{}
  

\begin{document}
\maketitle

\hypertarget{overview}{%
\subsection{OVERVIEW}\label{overview}}

This exercise accompanies the lessons in Environmental Data Analytics on
Data Exploration.

\hypertarget{directions}{%
\subsection{Directions}\label{directions}}

\begin{enumerate}
\def\labelenumi{\arabic{enumi}.}
\tightlist
\item
  Change ``Student Name'' on line 3 (above) with your name.
\item
  Work through the steps, \textbf{creating code and output} that fulfill
  each instruction.
\item
  Be sure to \textbf{answer the questions} in this assignment document.
\item
  When you have completed the assignment, \textbf{Knit} the text and
  code into a single PDF file.
\item
  After Knitting, submit the completed exercise (PDF file) to the
  dropbox in Sakai. Add your last name into the file name (e.g.,
  ``Salk\_A03\_DataExploration.Rmd'') prior to submission.
\end{enumerate}

The completed exercise is due on Tuesday, January 28 at 1:00 pm.

\hypertarget{set-up-your-r-session}{%
\subsection{Set up your R session}\label{set-up-your-r-session}}

\begin{enumerate}
\def\labelenumi{\arabic{enumi}.}
\tightlist
\item
  Check your working directory, load necessary packages (tidyverse), and
  upload two datasets: the ECOTOX neonicotinoid dataset
  (ECOTOX\_Neonicotinoids\_Insects\_raw.csv) and the Niwot Ridge NEON
  dataset for litter and woody debris
  (NEON\_NIWO\_Litter\_massdata\_2018-08\_raw.csv). Name these datasets
  ``Neonics'' and ``Litter'', respectively.
\end{enumerate}

\begin{Shaded}
\begin{Highlighting}[]
\KeywordTok{getwd}\NormalTok{()}
\end{Highlighting}
\end{Shaded}

\begin{verbatim}
## [1] "C:/Users/26059/OneDrive/Desktop/ENV 872 R/Yang_ENV872"
\end{verbatim}

\begin{Shaded}
\begin{Highlighting}[]
\KeywordTok{library}\NormalTok{(tidyverse)}
\NormalTok{Neonics <-}\StringTok{ }\KeywordTok{read.csv}\NormalTok{(}\StringTok{"./Data/Raw/ECOTOX_Neonicotinoids_Insects_raw.csv"}\NormalTok{)}
\NormalTok{Litter <-}\StringTok{ }\KeywordTok{read.csv}\NormalTok{(}\StringTok{"./Data/Raw/NEON_NIWO_Litter_massdata_2018-08_raw.csv"}\NormalTok{)}
\end{Highlighting}
\end{Shaded}

\hypertarget{learn-about-your-system}{%
\subsection{Learn about your system}\label{learn-about-your-system}}

\begin{enumerate}
\def\labelenumi{\arabic{enumi}.}
\setcounter{enumi}{1}
\tightlist
\item
  The neonicotinoid dataset was collected from the Environmental
  Protection Agency's ECOTOX Knowledgebase, a database for ecotoxicology
  research. Neonicotinoids are a class of insecticides used widely in
  agriculture. The dataset that has been pulled includes all studies
  published on insects. Why might we be interested in the ecotoxicologoy
  of neonicotinoids on insects? Feel free to do a brief internet search
  if you feel you need more background information.
\end{enumerate}

\begin{quote}
Answer: Neonicotinoid pesticides are widely applied to agriculture land.
The residue of neonicotinoid may post risk to non-target beneficial
insects.Insects are one of important parts of a healthy ecosystem.
Therefore, we might interested in the ecotoxicology of this chemical for
ecosystem health and balance.
\end{quote}

\begin{enumerate}
\def\labelenumi{\arabic{enumi}.}
\setcounter{enumi}{2}
\tightlist
\item
  The Niwot Ridge litter and woody debris dataset was collected from the
  National Ecological Observatory Network, which collectively includes
  81 aquatic and terrestrial sites across 20 ecoclimatic domains. 32 of
  these sites sample forest litter and woody debris, and we will focus
  on the Niwot Ridge long-term ecological research (LTER) station in
  Colorado. Why might we be interested in studying litter and woody
  debris that falls to the ground in forests? Feel free to do a brief
  internet search if you feel you need more background information.
\end{enumerate}

\begin{quote}
Answer: Insects constantly contact with litter and woody debris. Knowing
the concentration of chemical in litter and woody debris can help to
identify the chemical exposure of insects.
\end{quote}

\begin{enumerate}
\def\labelenumi{\arabic{enumi}.}
\setcounter{enumi}{3}
\tightlist
\item
  How is litter and woody debr is sampled as part of the NEON network?
  Read the NEON\_Litterfall\_UserGuide.pdf document to learn more. List
  three pieces of salient information about the sampling methods here:
\end{enumerate}

\begin{quote}
Answer: \emph{spatial sampling design }ramdom for the tower plots, and
target for the forest tower airshed *temporal sampling design
\end{quote}

\hypertarget{obtain-basic-summaries-of-your-data-neonics}{%
\subsection{Obtain basic summaries of your data
(Neonics)}\label{obtain-basic-summaries-of-your-data-neonics}}

\begin{enumerate}
\def\labelenumi{\arabic{enumi}.}
\setcounter{enumi}{4}
\tightlist
\item
  What are the dimensions of the dataset?
\end{enumerate}

\begin{Shaded}
\begin{Highlighting}[]
\KeywordTok{dim}\NormalTok{(Neonics)}
\end{Highlighting}
\end{Shaded}

\begin{verbatim}
## [1] 4623   30
\end{verbatim}

\begin{enumerate}
\def\labelenumi{\arabic{enumi}.}
\setcounter{enumi}{5}
\tightlist
\item
  Using the \texttt{summary} function, determine the most common effects
  that are studied. Why might these effects specifically be of interest?
\end{enumerate}

\begin{Shaded}
\begin{Highlighting}[]
\KeywordTok{summary}\NormalTok{(Neonics}\OperatorTok{$}\NormalTok{Effect)}
\end{Highlighting}
\end{Shaded}

\begin{verbatim}
##     Accumulation        Avoidance         Behavior     Biochemistry 
##               12              102              360               11 
##          Cell(s)      Development        Enzyme(s) Feeding behavior 
##                9              136               62              255 
##         Genetics           Growth        Histology       Hormone(s) 
##               82               38                5                1 
##    Immunological     Intoxication       Morphology        Mortality 
##               16               12               22             1493 
##       Physiology       Population     Reproduction 
##                7             1803              197
\end{verbatim}

\begin{quote}
Answer: Population and mortality are the most common effects that are
studied. Because both of them can reflect the acute effect of the
chemical, while other endpoints are reflecting chronic effect.
\end{quote}

\begin{enumerate}
\def\labelenumi{\arabic{enumi}.}
\setcounter{enumi}{6}
\tightlist
\item
  Using the \texttt{summary} function, determine the six most commonly
  studied species in the dataset (common name). What do these species
  have in common, and why might they be of interest over other insects?
  Feel free to do a brief internet search for more information if
  needed.
\end{enumerate}

\begin{Shaded}
\begin{Highlighting}[]
\KeywordTok{summary}\NormalTok{(Neonics}\OperatorTok{$}\NormalTok{Species.Common.Name)}
\end{Highlighting}
\end{Shaded}

\begin{verbatim}
##                          Honey Bee                     Parasitic Wasp 
##                                667                                285 
##              Buff Tailed Bumblebee                Carniolan Honey Bee 
##                                183                                152 
##                         Bumble Bee                   Italian Honeybee 
##                                140                                113 
##                    Japanese Beetle                  Asian Lady Beetle 
##                                 94                                 76 
##                     Euonymus Scale                           Wireworm 
##                                 75                                 69 
##                  European Dark Bee                  Minute Pirate Bug 
##                                 66                                 62 
##               Asian Citrus Psyllid                      Parastic Wasp 
##                                 60                                 58 
##             Colorado Potato Beetle                    Parasitoid Wasp 
##                                 57                                 51 
##                Erythrina Gall Wasp                       Beetle Order 
##                                 49                                 47 
##        Snout Beetle Family, Weevil           Sevenspotted Lady Beetle 
##                                 47                                 46 
##                     True Bug Order              Buff-tailed Bumblebee 
##                                 45                                 39 
##                       Aphid Family                     Cabbage Looper 
##                                 38                                 38 
##               Sweetpotato Whitefly                      Braconid Wasp 
##                                 37                                 33 
##                       Cotton Aphid                     Predatory Mite 
##                                 33                                 33 
##             Ladybird Beetle Family                         Parasitoid 
##                                 30                                 30 
##                      Scarab Beetle                      Spring Tiphia 
##                                 29                                 29 
##                        Thrip Order               Ground Beetle Family 
##                                 29                                 27 
##                 Rove Beetle Family                      Tobacco Aphid 
##                                 27                                 27 
##                       Chalcid Wasp             Convergent Lady Beetle 
##                                 25                                 25 
##                      Stingless Bee                  Spider/Mite Class 
##                                 25                                 24 
##                Tobacco Flea Beetle                   Citrus Leafminer 
##                                 24                                 23 
##                    Ladybird Beetle                          Mason Bee 
##                                 23                                 22 
##                           Mosquito                      Argentine Ant 
##                                 22                                 21 
##                             Beetle         Flatheaded Appletree Borer 
##                                 21                                 20 
##               Horned Oak Gall Wasp                 Leaf Beetle Family 
##                                 20                                 20 
##                  Potato Leafhopper         Tooth-necked Fungus Beetle 
##                                 20                                 20 
##                       Codling Moth          Black-spotted Lady Beetle 
##                                 19                                 18 
##                       Calico Scale                Fairyfly Parasitoid 
##                                 18                                 18 
##                        Lady Beetle             Minute Parasitic Wasps 
##                                 18                                 18 
##                          Mirid Bug                   Mulberry Pyralid 
##                                 18                                 18 
##                           Silkworm                     Vedalia Beetle 
##                                 18                                 18 
##              Araneoid Spider Order                          Bee Order 
##                                 17                                 17 
##                     Egg Parasitoid                       Insect Class 
##                                 17                                 17 
##           Moth And Butterfly Order       Oystershell Scale Parasitoid 
##                                 17                                 17 
## Hemlock Woolly Adelgid Lady Beetle              Hemlock Wooly Adelgid 
##                                 16                                 16 
##                               Mite                        Onion Thrip 
##                                 16                                 16 
##              Western Flower Thrips                       Corn Earworm 
##                                 15                                 14 
##                  Green Peach Aphid                          House Fly 
##                                 14                                 14 
##                          Ox Beetle                 Red Scale Parasite 
##                                 14                                 14 
##                 Spined Soldier Bug              Armoured Scale Family 
##                                 14                                 13 
##                   Diamondback Moth                      Eulophid Wasp 
##                                 13                                 13 
##                  Monarch Butterfly                      Predatory Bug 
##                                 13                                 13 
##              Yellow Fever Mosquito                Braconid Parasitoid 
##                                 13                                 12 
##                       Common Thrip       Eastern Subterranean Termite 
##                                 12                                 12 
##                             Jassid                         Mite Order 
##                                 12                                 12 
##                          Pea Aphid                   Pond Wolf Spider 
##                                 12                                 12 
##           Spotless Ladybird Beetle             Glasshouse Potato Wasp 
##                                 11                                 10 
##                           Lacewing            Southern House Mosquito 
##                                 10                                 10 
##            Two Spotted Lady Beetle                         Ant Family 
##                                 10                                  9 
##                       Apple Maggot                            (Other) 
##                                  9                                670
\end{verbatim}

\begin{quote}
Answer: The most common is honey bee, parasitic wasp, buff tailed
bumblebee, carniolan honey bee, bumble bee and Italian honeybee. The six
most commonly species are all pollinators. They got more attention than
other species because they are important pollinators for agriculture
crops.
\end{quote}

\begin{enumerate}
\def\labelenumi{\arabic{enumi}.}
\setcounter{enumi}{7}
\tightlist
\item
  Concentrations are always a numeric value. What is the class of
  Conc.1..Author. in the dataset, and why is it not numeric?
\end{enumerate}

\begin{Shaded}
\begin{Highlighting}[]
\KeywordTok{class}\NormalTok{(Neonics}\OperatorTok{$}\NormalTok{Conc.}\DecValTok{1}\NormalTok{..Author.)}
\end{Highlighting}
\end{Shaded}

\begin{verbatim}
## [1] "factor"
\end{verbatim}

\begin{Shaded}
\begin{Highlighting}[]
\KeywordTok{str}\NormalTok{(Neonics}\OperatorTok{$}\NormalTok{Conc.}\DecValTok{1}\NormalTok{..Author.)}
\end{Highlighting}
\end{Shaded}

\begin{verbatim}
##  Factor w/ 1006 levels "~10","~30/","~40/",..: 639 510 813 622 442 637 500 642 814 784 ...
\end{verbatim}

\begin{quote}
Answer: It is factor. Because it has non-numeric subjects and symbol
like slash..
\end{quote}

\hypertarget{explore-your-data-graphically-neonics}{%
\subsection{Explore your data graphically
(Neonics)}\label{explore-your-data-graphically-neonics}}

\begin{enumerate}
\def\labelenumi{\arabic{enumi}.}
\setcounter{enumi}{8}
\tightlist
\item
  Using \texttt{geom\_freqpoly}, generate a plot of the number of
  studies conducted by publication year.
\end{enumerate}

\begin{Shaded}
\begin{Highlighting}[]
\KeywordTok{ggplot}\NormalTok{(Neonics) }\OperatorTok{+}
\StringTok{  }\KeywordTok{geom_freqpoly}\NormalTok{(}\KeywordTok{aes}\NormalTok{(}\DataTypeTok{x =}\NormalTok{ Publication.Year),}\DataTypeTok{binwidth =} \DecValTok{1}\NormalTok{) }
\end{Highlighting}
\end{Shaded}

\includegraphics{A03_DataExploration_files/figure-latex/unnamed-chunk-6-1.pdf}

\begin{enumerate}
\def\labelenumi{\arabic{enumi}.}
\setcounter{enumi}{9}
\tightlist
\item
  Reproduce the same graph but now add a color aesthetic so that
  different Test.Location are displayed as different colors.
\end{enumerate}

\begin{Shaded}
\begin{Highlighting}[]
\KeywordTok{ggplot}\NormalTok{(Neonics) }\OperatorTok{+}
\StringTok{  }\KeywordTok{geom_freqpoly}\NormalTok{(}\KeywordTok{aes}\NormalTok{(}\DataTypeTok{x =}\NormalTok{ Neonics}\OperatorTok{$}\NormalTok{Publication.Year, }\DataTypeTok{color =}\NormalTok{ Neonics}\OperatorTok{$}\NormalTok{Test.Location ),}\DataTypeTok{binwidth =} \DecValTok{1}\NormalTok{) }
\end{Highlighting}
\end{Shaded}

\includegraphics{A03_DataExploration_files/figure-latex/unnamed-chunk-7-1.pdf}

Interpret this graph. What are the most common test locations, and do
they differ over time?

\begin{quote}
Answer:
\end{quote}

\begin{enumerate}
\def\labelenumi{\arabic{enumi}.}
\setcounter{enumi}{10}
\tightlist
\item
  Create a bar graph of Endpoint counts. What are the two most common
  end points, and how are they defined? Consult the ECOTOX\_CodeAppendix
  for more information.
\end{enumerate}

\begin{Shaded}
\begin{Highlighting}[]
\KeywordTok{ggplot}\NormalTok{(Neonics, }\KeywordTok{aes}\NormalTok{(}\DataTypeTok{x =}\NormalTok{ Neonics}\OperatorTok{$}\NormalTok{Endpoint)) }\OperatorTok{+}
\StringTok{  }\KeywordTok{geom_bar}\NormalTok{()}
\end{Highlighting}
\end{Shaded}

\includegraphics{A03_DataExploration_files/figure-latex/unnamed-chunk-8-1.pdf}

\begin{quote}
Answer: The most common end points are LOEL and NOEL.LOEL is the lowest
observed adverse effect level. NOEL is the highest no observable effect
level.
\end{quote}

\hypertarget{explore-your-data-litter}{%
\subsection{Explore your data (Litter)}\label{explore-your-data-litter}}

\begin{enumerate}
\def\labelenumi{\arabic{enumi}.}
\setcounter{enumi}{11}
\tightlist
\item
  Determine the class of collectDate. Is it a date? If not, change to a
  date and confirm the new class of the variable. Using the
  \texttt{unique} function, determine which dates litter was sampled in
  August 2018.
\end{enumerate}

\begin{Shaded}
\begin{Highlighting}[]
\KeywordTok{class}\NormalTok{(Litter}\OperatorTok{$}\NormalTok{collectDate)}
\end{Highlighting}
\end{Shaded}

\begin{verbatim}
## [1] "factor"
\end{verbatim}

\begin{Shaded}
\begin{Highlighting}[]
\NormalTok{Litter}\OperatorTok{$}\NormalTok{collectDate <-}\StringTok{ }\KeywordTok{as.Date}\NormalTok{(Litter}\OperatorTok{$}\NormalTok{collectDate,}\DataTypeTok{format=}\StringTok{"%Y-%m-%d"}\NormalTok{) }
\KeywordTok{class}\NormalTok{(Litter}\OperatorTok{$}\NormalTok{collectDate)}
\end{Highlighting}
\end{Shaded}

\begin{verbatim}
## [1] "Date"
\end{verbatim}

\begin{Shaded}
\begin{Highlighting}[]
\KeywordTok{unique}\NormalTok{(Litter}\OperatorTok{$}\NormalTok{collectDate)}
\end{Highlighting}
\end{Shaded}

\begin{verbatim}
## [1] "2018-08-02" "2018-08-30"
\end{verbatim}

\begin{enumerate}
\def\labelenumi{\arabic{enumi}.}
\setcounter{enumi}{12}
\tightlist
\item
  Using the \texttt{unique} function, determine how many plots were
  sampled at Niwot Ridge. How is the information obtained from
  \texttt{unique} different from that obtained from \texttt{summary}?
\end{enumerate}

\begin{Shaded}
\begin{Highlighting}[]
\KeywordTok{unique}\NormalTok{(Litter}\OperatorTok{$}\NormalTok{plotID)}
\end{Highlighting}
\end{Shaded}

\begin{verbatim}
##  [1] NIWO_061 NIWO_064 NIWO_067 NIWO_040 NIWO_041 NIWO_063 NIWO_047
##  [8] NIWO_051 NIWO_058 NIWO_046 NIWO_062 NIWO_057
## 12 Levels: NIWO_040 NIWO_041 NIWO_046 NIWO_047 NIWO_051 ... NIWO_067
\end{verbatim}

\begin{Shaded}
\begin{Highlighting}[]
\KeywordTok{summary}\NormalTok{(Litter}\OperatorTok{$}\NormalTok{plotID)}
\end{Highlighting}
\end{Shaded}

\begin{verbatim}
## NIWO_040 NIWO_041 NIWO_046 NIWO_047 NIWO_051 NIWO_057 NIWO_058 NIWO_061 
##       20       19       18       15       14        8       16       17 
## NIWO_062 NIWO_063 NIWO_064 NIWO_067 
##       14       14       16       17
\end{verbatim}

\begin{quote}
Answer: 12 plots. The information obtained from unique function does not
include the count of the same plot. However, It gives the number of the
unique plots and has a rank of the plot code.
\end{quote}

\begin{enumerate}
\def\labelenumi{\arabic{enumi}.}
\setcounter{enumi}{13}
\tightlist
\item
  Create a bar graph of functionalGroup counts. This shows you what type
  of litter is collected at the Niwot Ridge sites. Notice that litter
  types are fairly equally distributed across the Niwot Ridge sites.
\end{enumerate}

\begin{Shaded}
\begin{Highlighting}[]
\KeywordTok{ggplot}\NormalTok{(Litter, }\KeywordTok{aes}\NormalTok{(}\DataTypeTok{x =}\NormalTok{ Litter}\OperatorTok{$}\NormalTok{functionalGroup)) }\OperatorTok{+}
\StringTok{  }\KeywordTok{geom_bar}\NormalTok{()}
\end{Highlighting}
\end{Shaded}

\includegraphics{A03_DataExploration_files/figure-latex/unnamed-chunk-11-1.pdf}

\begin{enumerate}
\def\labelenumi{\arabic{enumi}.}
\setcounter{enumi}{14}
\tightlist
\item
  Using \texttt{geom\_boxplot} and \texttt{geom\_violin}, create a
  boxplot and a violin plot of dryMass by functionalGroup.
\end{enumerate}

\begin{Shaded}
\begin{Highlighting}[]
\KeywordTok{ggplot}\NormalTok{(Litter) }\OperatorTok{+}
\StringTok{  }\KeywordTok{geom_boxplot}\NormalTok{(}\KeywordTok{aes}\NormalTok{(}\DataTypeTok{x =}\NormalTok{ Litter}\OperatorTok{$}\NormalTok{functionalGroup, }\DataTypeTok{y =}\NormalTok{ dryMass))}
\end{Highlighting}
\end{Shaded}

\includegraphics{A03_DataExploration_files/figure-latex/unnamed-chunk-12-1.pdf}

\begin{Shaded}
\begin{Highlighting}[]
\KeywordTok{ggplot}\NormalTok{(Litter) }\OperatorTok{+}
\StringTok{  }\KeywordTok{geom_violin}\NormalTok{(}\KeywordTok{aes}\NormalTok{(}\DataTypeTok{x =}\NormalTok{ Litter}\OperatorTok{$}\NormalTok{functionalGroup, }\DataTypeTok{y =}\NormalTok{ dryMass), }
                                \DataTypeTok{scale=}\StringTok{"area"}\NormalTok{)}
\end{Highlighting}
\end{Shaded}

\includegraphics{A03_DataExploration_files/figure-latex/unnamed-chunk-12-2.pdf}

Why is the boxplot a more effective visualization option than the violin
plot in this case?

\begin{quote}
Answer: The dry mass between different functional groups vary too much.
\end{quote}

What type(s) of litter tend to have the highest biomass at these sites?

\begin{quote}
Answer: Needles
\end{quote}


\end{document}
